% LaTeX resume using res.cls
\documentclass[margin]{res}
%\usepackage[scaled]{helvet}
\usepackage{helvetica} % uses helvetica postscript font (download helvetica.sty)
\usepackage{enumitem}
\setlist{nolistsep}
\usepackage[none]{hyphenat}
%\usepackage{newcent}   % uses new century schoolbook postscript font
\setlength{\textwidth}{5.1in} % set width of text portion
\usepackage{fancyhdr}
\usepackage{calc}
\fancypagestyle{plain}{%
   \fancyhf{}
   \renewcommand{\headrulewidth}{0pt}
   %\headsep = 1.2cm
   \setlength{\textheight}{650pt}
   \fancyheadoffset[HL]{4.7cm}
   \fancyfootoffset[L]{3.3cm}
   \fancyfoot[L]{Kyle R. Timins}
   \fancyfoot[R]{Page \thepage}}

\fancypagestyle{empty}{%
   \fancyhf{}
   \headsep = 0cm
   \renewcommand{\headrulewidth}{0pt}
   \fancyfootoffset[L]{3.3cm}
   \fancyfoot[L]{Kyle R. Timins}
   \fancyfoot[R]{Page \thepage}}

\begin{document}
\pagestyle{plain}
\thispagestyle{empty}
%% Center the name over the entire width of resume:
\moveleft.5\hoffset\centerline{\large\bf Kyle R. Timins}
%% Draw a horizontal line the whole width of resume:
\moveleft\hoffset\vbox{\hrule width\resumewidth height 1pt}\smallskip
%% address begins here
%% Again, the address lines must be centered over entire width of resume:
\moveleft.5\hoffset\centerline{161 Oxford Drive}
\moveleft.5\hoffset\centerline{South Windsor, CT 06074}
\moveleft.5\hoffset\centerline{(860) 212-2254}
\moveleft.5\hoffset\centerline{KyleRTimins@gmail.com}


\begin{resume}

   \section{OBJECTIVE} %\#\#\# CHANGE ME \#\#\#
   To secure a full time position utilizing my $\tilde{}$ five years of workforce experience
   along with my own personal learned knowledge.


   \section{EDUCATION}
   {\sl Bachelor of Science,} 
   Computer Software Engineering,
   Vermont Technical College, Randolph Center, VT,\\
   Graduated May 2014 \\
   \vspace{-.5cm}

   \begin{tabbing}
   \section{COMPUTER \\ SKILLS}
      {\sl Languages:} 
      Powershell, C\#, VB6, Bash, \LaTeX, Perl, VB.Net, T-SQL, XML/XPath, \\ 
      AutoHotKey \\
      {\sl Operating Systems:} 
      Windows, GNU/Linux, Unix \\
      {\sl Programs:} 
      Vim \\
      {\sl Exposure To:} 
      Ada, Computer security, FreeBSD, CLISP, Java, Scheme, MVC \\
      {\sl Actively Learning:}
      ASP.NET Core MVC, Scheme
   \end{tabbing}

   \section{EXPERIENCE} 
   {\sl Senior Software Developer} \hfill Jun 2014 -- Current \\
   {\sl Software Engineer \&\& Senior Software Engineer} \hfill // Previous Titles \\
   Insurity, Hartford, CT\\\
   Multiple Areas
   \begin{itemize}  %\itemsep -2pt %reduce space between items
      \item Worked on Backend of thirteen Property \& Causality Lines of Business
      \begin{itemize}
      	 \item Maintained (Bug fixes) both Legacy and the new ``Evolution'' products
      	 \item Created new functionality via both Customer enhancements and \\
      	    Regulatory enhancements
      	 \item Refactored codebase to increase maintainability, readability, 
      	    and bring them more inline with SOLID principles
      	 \item {\sl Parts of the Backend System:} Rating, Mapping, Statistical Coding, 
      	    Policy Forms Generation
      	 \item {\sl Lines of Business:} Property Casualty, Business Owners, ISO Inland \\
      	     Marine, AAIS Inland Marine, Ocean Marine, Boiler and Machinery, \\
      	     Commercial Output, Capital Assets, Cargo, and Farm
      	     \begin{itemize}
      	        \item Was part of the original development team to bring up Farm and 
      	           Cargo
      	        \item Learned parts of the General Liability and Auto Lines to be able 
      	           to help others that were a part of those teams
      	     \end{itemize}
      \end{itemize}
      \item Worked on interfaces that created the feed to downstream Reporting 
         Systems like WINS and Custom Systems
         \begin{itemize}
            \item Became the sole Subject Matter Expert (SME) in these areas when the 
               previous SME left less than a month after I started learning these areas
            \item Worked with our internal WINS team and Clients' WINS teams to gain a 
               better understanding and knowledge of the WINS system itself
            \item Worked with my manager, at that time, on an initiative to \\
            distribute 
               knowledge and workload of this to Line of Business teams where 
               applicable. 
               Tasks of this initiative included:
               \begin{itemize}
                  \item Setup training classes/presentations 
                  \item Work one on one with those who would be performing harder tasks, 
                     such as bringing up new Client or Lines from scratch
                  \item Maintain an ``Open Door'' policy with helping others
               \end{itemize}
         \end{itemize}
      \item Worked on creation of XML that passed information from our Policy system 
         to the Billing system.
         \begin{itemize}
            \item Became a ``liaison'' between our Policy system team and the Billing 
               system team to increase communication and efficiency, including creating 
               proper procedures on how the two systems and teams work together
            \item Currently working on an effort to bring the XML created into proper \\
               compliance with XML namespaces and namespace prefixes
         \end{itemize}
      \item Helped create automated systems, behind the scenes, that made labor heavy 
         and mundane tasks into automated processes that went beyond current \\
         processes.
         \begin{itemize}
            \item Created an automated system, in Powershell, that increased reporting 
               and visibility of errors occurring in hosted environments utilizing our 
               current error system that would only create a Zip file of log files and 
               entry in a database. This worked as a ``second line of defense'' in 
               conjunction with standard environment monitoring systems.
               \begin{itemize}
                  \item Was designed and implemented as a JSON based configured system 
                     allowing for a variety of different ``run reasons'' without having 
                     to change the program itself.
                  \item Allowed the user to pick from a variety of different predetermined 
                     categories of errors that were known to happen or define their own set 
                     of exception messages or error stacks, utilizing SQL Server wild cards, 
                     including the null set to get back all errors. A set of ``not'' exceptions 
                     can also be defined to exclude certain errors, also utilizing SQL Server 
                     wild cards, from being reported in the run. 
                     Multiple sets could be defined in a config file to give a more 
                     encompassing view of the system status. 
                     Further fine tuning could be done by giving timespans to look back, a 
                     minimum count threshold, per item in the set, the system that threw 
                     the error, and the Line(s) of Business associated with the error.
                  \item The config file allowed for defining a set, one or more, Client and 
                     environment pair(s) to define which database to gather information from
                  \item CSV, JSON, and/or XML files could be created, as defined in the  
                     config file, to allow for further reporting of errors.
                  \item An email could also be created containing the error information, with 
                     granular tuning of information included defined by error set. Reporting 
                     files could be attached to the email. (Zip files were originally attached 
                     as well, but functionally was removed due to email size limitation and 
                     replaced with a link to where to get a zip of zips.) A set of email 
                     recipients could also be 
                     defined based on a defined service pack (aka. sprint) or a specified date range 
                     or timespan from a specified start date.
               \end{itemize}
            \item Created an automated reporting and notification system of scheduled assignments. 
               \begin{itemize}
                  \item The previous process involved creating a spreadsheet of assignments that 
                     remained ``stagnant'' after the start of the service pack (aka. sprint) it was 
                     for. After the system was put through a trial run, it was implemented as official 
                     procedure for assignment schedule reporting and notification.
                  \item Was originally created in Powershell to utilize functionality I created in 
                     other scripts. It was later ported over to a full C\# application.
                  \item Was designed and implemented as JSON based configured system allowing for a 
                      variety of different ``run reasons'' without having to modify the program itself.
                      This was used in conjunction with command line arguments to define the tasks
                      desired for that run. Command line arguments also allowed for overwriting the
                      definitions in the config file.
                  \item Was originally a command line program only. However, a GUI was later added to
                     increase usability of ``one off'' runs.
                  \item It utilized a wild card based system to determine what spreadsheet to read in
                     based on a lookup table of conversion between a date span and the defined service
                     pack numbering. Naming could be changed in the config file.
                  \item It would read in the spreadsheet of assignments and call out to the Assignment
                     Management System via API to get up-to-date \\
                     information on each assignment including
                     current resource assigned and estimated \& completed hours. This could be scheduled
                     via \\
                     Windows Scheduled Tasks or run manually on-demand. A CSV file, in the Powershell 
                     version, or a JSON file, in the C\# version, would be created to store information 
                     collected so that information reporting could be performed without an update of data.
                  \item An XLSX file could be created to allow for reporting and analysis. This was later
                     utilized for Assignees to be able to state their current status of their assignments.
                     (i.e. Currently being worked on, delayed, will not make the scheduled due date, 
                     etc...)
                  \item Emails could be created to be sent to the Assignees with their current list, to
                     Managers of all assignments for those under them grouped by Assignee, and Account
                     Managers of all assignments for Clients they are responsible for grouped by Client 
                     and Assignee. CCed recipients could be specified, along with a ``Do not send to'' 
                     list in case people didn't want to be included on the emails. As well, a CSV file 
                     was utilized defining a lookup of recipients that should be automatically added if
                     a defined recipient was already on the email.
               \end{itemize}
         \end{itemize}
      \item Spearheaded a project to upgrade to MSXML6 in our legacy system. This utilized our official
         build system, along with a custom built Powershell build script, and our developer update system
         that I modified to allow for distributing the DLLs created from a build machine to a test machine
         that existed outside
         of our standard build and update system. My role in this was not only to do software upgrades, 
         but also become a ``one man'' DevOps team in a siloed environment.
      \item Various, random tasks that covered all parts of the company.
   \end{itemize}

   {\sl IT Summer Intern} \hfill Jun 2013 -- Aug 2013 \\
   Travelers Ins., Hartford, CT\\\
   OAI-SYS Department -- Information - Systems Security
   \begin{itemize}  %\itemsep -2pt %reduce space between items
      \item Created Perl 5 scripts to perform metrics on
         compliance
      \item Created and maintained database of Portable
         Media Exceptions
      \item Created basic list of endpoint security
      \item Capstone on current and potential use of mobile
         devices in Travelers' Claim Department
   \end{itemize}

   {\sl IT Summer Intern} \hfill Jun 2012 -- Aug 2012 \\
   Travelers Ins., Hartford, CT\\\
   OAI-SYS Department -- Service Management
   \begin{itemize} %\itemsep -2pt %reduce space between items
      \item Created Perl 5 scripts to increase efficiency and
         automation of Server Audits
   \end{itemize}
   \section{EXPERIENCE \\ CONT.}
   \begin{itemize}
      \item Performed Server Audit
      \item Headed XMatters add-on research to increase escalation
         efficiency
      \item Worked on N05/C03 patches for BMC TM ART monitoring
         system
      \item Capstone on current and potential use of
         automation in Travelers
   \end{itemize}
   
   {\sl IT Summer Intern} \hfill  May 2011 -- Aug 2011 \\
   Travelers Ins., Hartford, CT\\\
   OAI-SYS Department -- Advanced Technologies
   \begin{itemize}   %\itemsep -2pt %reduce space between items
      \item Created Perl 5 scripts to increase efficiency and
         automation
      \item Fixed web applications to allow for cross-browser
         compatibility
         \begin{itemize} %\itemsep -2pt %reduce space between items
            \item Allowed custom tool tip pop-up to be used in non
               Internet Explorer \\
               browsers
            \item Create copy from browser work-around for non
               Internet Explorer \\
               browsers
         \end{itemize}
      \item Remotely reinstalled firewall device operating
         system and created written \\
         documentation
      \item Organized retired Cisco networking equipment for
         recycling
      \item Capstone on potential use of open source software
         in Travelers
   \end{itemize}

\end{resume}
\end{document}






